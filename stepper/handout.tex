% arara: lualatex
% arara: lualatex
% Typeset using lualatex

\documentclass{../tutorial}
\usepackage{wrapfig}

\title{Turning Stepper motors with MicroPython on \abbr{ESP32}}

\begin{document}

\begin{enumerate}

\item
    You will need a set-up environment and an \abbr{ESP32} board
    with a \abbr{ULN2803} chip.

\section{I/O Expander}

XXX

\section{Some theory on stepper motors}

\item
    A \emph{stepper motor} is a device that rotate in discrete steps.

    Steppers allow precise continuous movement: in printers, 3D printers
    and plotters, automated window blinds, car window wipers and so on.

    Steppers work by attracting the rotating center, the \emph{rotor}, to
    strategically placed electromagnets around it.

    \begin{figure}[h]
    \foreach \angle in {0, 90, 180, 270}{
        \foreach \n in {1, 2} {
            \begin{tikzpicture}
                \node[rotate=-\angle] {
                    \includegraphics[width=0.1\textwidth]{stepper\n}
                };
            \end{tikzpicture}
        }
    }
    \caption{Turning a stepper motor. Active electromagnets are highlighted red.}
    \end{figure}

    \begin{comment}
        This is simplified for clarity.
        There are many variations and improvements on this basic principle.
    \end{comment}

    There are a few things to watch out for.

    \begin{itemize}
    \item
    To allow the motor to turn, you'll need to wait about $1 \si{\milli\second}$
    after setting the electromagnets.

    \item
    Eight steps per revolution isn't a lot, so the motor has some gears inside
    to make the axle turn more slowly.
    It takes 512 full turns of the rotor to turn the axle around once (360°).

    \item
    Each of the 4 magnets needs to be controlled by a Pin.
    That's a lot of pins (especially if you want more motors),
    so we'll use an IO expander.

    \item
    We can't drive a motor with I/O pins directly, as
    the current needed to turn a motor could damage the microcontroller.
    We have a specialized chip for that: the \abbr{ULN2803}
    \emph{Darlington array}.
    It has 8 binary logic (“pin”) inputs and 8 corresponding high-current
    outputs driven by a dedicated power source (batteries).
    But software-wise, it works as if the motor was connected directly to
    the I/O expander.
    \end{itemize}

\section{Turn a stepper motor}

\item
    Write a program to drive the motor.
    Use pins 0-4 on the I/O expander for this.

    You'll need to output the following bit patterns:

    \begin{lstlisting}
    0001
    0011
    0010
    0110
    0100
    1100
    1000
    1001
    \end{lstlisting}

    To make this visible, delay for $100 \si{\milli\second}$ after setting each
    pattern.

    Let the booth staff check the result and give you a stepper motor
    to play with.

\item
    Plug in the motor and reduce the delay to $100 \si{\milli\second}$.

    Turn the motor exactly $90 \si{\degree}$ – that is, cycle through the bit
    patterns $128$ times.

\item
    Turn $90 \si{\degree}$ in the opposite direction.

\item
    Look at the 3D printer at the next booth to see how a stepper's precise
    turning is converted to linear motion.

\item
    Make the motor turn continuously.

    Get a second motor, and make it turn at $\frac23$ speed of the first one.

    How would this kind of behavior be useful in a 3D printer?

\item
     Please return the gadgets when you're done experimenting.

\end{enumerate}
\end{document}
