% arara: lualatex
% arara: lualatex
% Typeset using lualatex

\documentclass{../tutorial}

\title{Turning servos with MicroPython on \abbr{ESP32}}

\begin{document}

\begin{enumerate}

\item
    You will need a set-up environment and an \ESP board
    with a servo motor.

\section{Some theory}

\item
    A \emph{servo motor} is a device that can be turned to a specific angle.

    They're used to control airplane flaps, robot arms,
    camera auto-focus mechaisms, and so on.

    Hobby servos are controlled by \emph{timed pulses}: a controlling pin is
    set to `1` for a short time, like this:

    \begin{figure}[h]
        \centering
        \squarevave{2/2.5}
    \end{figure}

    \begin{figure}[h]
        \centering
        \squarevave{2/4}
    \end{figure}

    The servo's electronics measure the width (time) of the pulse and
    turn to an appropriate position.

    After some time, the servo turns off.
    To reliably keep a given angle,
    the pulse needs to be repeated like this:

    \begin{figure}[h]
        \centering
        \squarevave{0/.5, 3/3.5, 6/6.5, 9/9.5, 12/12.5}
    \end{figure}

    You've seen this before! That's a square wave.
    There are nice highter-level libraries for servos,
    but simple PWM will do just fine today.
    You just need to set the proper frequency and duty cycle.

    It's possible to calculate these parameters yourself, but we'we done that
    for you.\footnote{
        Our servo needs a pulse width of
        $1 \si{\milli\second}$ to $2 \si{\milli\second}$.
        With the PWM at $50 \si{\hertz}$, that's:

        \begin{itemize}
        \item minimum:
            ${0.001 \si{\second}} \times {50 \si{\hertz}} \times 100\% = 5\%$
        \item maximum:
            ${0.002 \si{\second}} \times {50 \si{\hertz}} \times 100\% = 10\%$
        \end{itemize}

        Converted to the $0...1024$ duty range, that's roughly $50$ to $100$.
        \emph{Roughly} – these cheap servos aren't too exact.
    }
    Use `freq=50` and `duty` between `50` and `100`.


\section{Controlling the servo}

\item
    Create a PWM on pin \#17:

    \begin{lstlisting}
    from machine import Pin, PWM
    pwm = PWM(Pin(17, Pin.OUT), freq=50, duty=75)
    \end{lstlisting}

    Then, set the duty to turn the servo:

    \begin{lstlisting}
    pwm.duty(50)   # left
    pwm.duty(75)   # middle
    pwm.duty(100)  # right
    \end{lstlisting}

    To control where “middle” is, pull off the white servo arm and re-attach it
    in a different position.

\item
    Experiment! But if you decide go beyond these parameters and hear an
    unusual buzzing sound from the servo, please return to `duty(50)` or reset
    the board to avoid damaging the servo.

\item
    Make a metronome!

\item
    How much time does it take for the servo to turn 90°?

    \begin{comment}
        The servo doesn't give any notification when it's done turning.
        You'll need a watch.
    \end{comment}

\clearpage

% XXX: Temperature gauge

\item
    Please return the board when you're done experimenting.


\end{enumerate}
\end{document}
