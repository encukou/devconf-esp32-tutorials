% arara: lualatex
% arara: lualatex
% Typeset using lualatex

\documentclass{../tutorial}
\usepackage{wrapfig}

\title{Wi-Fi Communication with the \abbr{ESP32}}

\begin{document}

    The \abbr{ESP32}'s predecessor, \abbr{ESP8266}, was designed as a dedicated
    Wi-Fi chip,
    and the \abbr{ESP32} is also quite a capable Wi-Fi device.
    Let's take advantage of that!

    You will need a set-up environment and an \abbr{ESP32} board.

\begin{enumerate}

\item
    The conference's peculiar Wi-Fi setup is quite inconvenient for
    microcontrollers, but we can make our own network!

    Turn the \abbr{ESP32}'s into an access point (`AP_IF`):

    \begin{lstlisting}[escapeinside=<>]
    import network
    ap = network.WLAN(network.AP_IF)
    ap.active(True)
    ap.config(essid='<\emph{«your own name»}>', password='<\emph{«something secret»}>',
              channel=<\emph{«some small number»}>, authmode=network.AUTH_WPA2_PSK)
    \end{lstlisting}

    \begin{comment}
        Ideally, run `nmcli d wifi` on your computer to see occupied
        Wi-Fi channels, and pick a channel you don't see.
    \end{comment}

    The `ifconfig` method can give you interface parameters – IP address,
    subnet mask, gateway and DNS server:

    \begin{lstlisting}
    print(ap.ifconfig())
    \end{lstlisting}

    Now, connect to your Wi-Fi using your computer.
    Can you ping the device?

\item
    On your computer, figure out your new IP address using:

    \begin{lstlisting}[language=bash]
    ip -4 a
    \end{lstlisting}

    Keep it in mind for later.

\section{HTTP Requests}

\item
    Let's try to download a file from your computer to the \abbr{ESP32}
    using HTTP.

    On your computer, create an new directory and put a small text file in it.
    Then, `cd` to the directory and start a HTTP server on port 8000
    to share the file:

    \begin{lstlisting}[language=bash]
    python3 -m http.server
    \end{lstlisting}

    Meanwhile, in MicroPython, import $\mu$requests – a minimal
    reimplementation of Requests, a HTTP client library:

    \begin{lstlisting}
    import urequests
    \end{lstlisting}

    Then, request your file!

    \begin{lstlisting}[escapeinside=<>]
    response = urequests.get('http://<\emph{«your.ip.address»}>:8000/<\emph{filename.txt}>')
    print(response.status_code)
    \end{lstlisting}

    If that gives you an `EHOSTUNREACH` error, you might need to lower your
    firewall on this port.
    For Fedora, punch a temporary hole using this command (the hole will be
    closed on reboot, or with `firewall-cmd --reload`):

    \begin{lstlisting}
    firewall-cmd --add-port=8000/tcp
    \end{lstlisting}

    If you get a 200 (OK) status, see what the response text is:

    \begin{lstlisting}
    print(response.text)
    \end{lstlisting}

    Please keep in mind the HTTP connection is not encrpyted.
    With some effort, anyone can impersonate your server.

\section{UDP}

\item
    The HTTP(s) protocol is not very suitable for microcontrollers.
    There are better ones available, such as MQTT (for which there are
    MicroPython libraries available).

    We, however, will use something even more low-level: UDP,
    an unreliable, connection-less unencrypted protocol.

    The same code should work both on your computer and on the \abbr{ESP32}.
    Decide which one will send and which will receive data.
    On the computer, you'll need to open up your firewall; for Fedora that's
    `firewall-cmd --add-port=48334/udp`.

    To \emph{send} data, create a `SOCK_DGRAM` (UDP) socket and use `sendto`
    with a message and destination address.
    You need to provide a unique port number larger than 1000 – say, 48334
    for $\mu$Python.

    Here is example code for sending:

    \begin{lstlisting}[escapeinside=<>]
    import time
    import socket

    address = ('<\emph{«destination IP address»}>', <\emph{48334}>)

    sending_socket = socket.socket(socket.AF_INET, socket.SOCK_DGRAM)

    while True:
        message = b'Hello'
        print('Sending', message)
        sending_socket.sendto(message, address)
        time.sleep(1/2)
    \end{lstlisting}

    To \emph{receive} data \emph{bind} a `SOCK_DGRAM` socket to a particular
    port.
    This way, on a computer, the operating system knows which program to send
    data to.
    You can also set a timeout, so if no message comes, your program won't hang.

    The IP address is not needed for receiving.

    \begin{lstlisting}[escapeinside=<>]
    import socket
    import time

    address = ('', <48334>)

    receiving_socket = socket.socket(socket.AF_INET, socket.SOCK_DGRAM)
    receiving_socket.bind(address)
    receiving_socket.settimeout(1/2)

    while True:
        try:
            message = receiving_socket.recv(1024)
        except OSError:
            print('No message')
        else:
            print('Got', message)
    \end{lstlisting}

    On the \abbr{ESP32}, where your program is the only thing running,
    the sender can send to \emph{all} devices in its local network with the
    \emph{broadcast address} `255.255.255.255`.
    Also, you don't need to call `bind` when receiving.

    Keep in mind the UDP connection is not encrpyted or authenticated.
    Anyone can send anything and you won't even know who sent which message.

\section{Connecting to existing Wi-Fi}

\item
    If there is an existing Wi-Fi network, you can connect to it as a “Station”
    instead of creating an Access Point.
    Use `STA_IF` in place of `AP_IF`, and `connect` in place of `config`.

    MicroPython establishes the connection in the background.
    Usually, you'll want to wait until `isconnected()` reports `True`.

    \begin{lstlisting}[escapeinside=<>]
    import network
    from time import sleep

    wlan = network.WLAN(network.STA_IF)
    wlan.active(True)
    wlan.connect('<\emph{«network name»}>', '<\emph{«password»}>')

    while not wlan.isconnected():
        print('Connecting....')
        sleep(1)

    print(wlan.ifconfig())  # Our IP address and other interface parameters
    \end{lstlisting}

\item
    There should be a joystick broadcasting its state over Wi-Fi.
    Ask the staff for details.

    Take one of your other programs (with LEDs, motors, etc.)
    and let it react to the joystick's position.

\item
    Get a second \abbr{ESP32} and make it impersonate the joystick,
    sending different data.

\item
    Please return the gadgets when you're done experimenting.

\end{enumerate}
\end{document}

    \begin{lstlisting}
