% arara: lualatex
% arara: lualatex
% Typeset using lualatex

\documentclass[a4paper,10pt]{article}
\usepackage[utf8]{luainputenc}
\usepackage[pdfborder={0 0 0}]{hyperref}

\usepackage[english]{babel}

\usepackage{fouriernc}
\usepackage{tgpagella}
\usepackage{xcolor}
\usepackage[activate=true]{microtype}
\usepackage{amssymb}
\usepackage{fontspec}
\usepackage{titling}
\usepackage{titlesec}

\usepackage{calc}
\usepackage{multicol}
\usepackage{array}
\usepackage{enumitem}
\usepackage{tikz}
\usepackage{siunitx}
\usepackage{colortbl}
\usepackage[a4paper,margin=2cm,top=1cm]{geometry}

\setmainfont[Numbers=OldStyle]{TeX Gyre Pagella}

\definecolor{plhome}{HTML}{CCEEFF}
\definecolor{silver}{HTML}{DDDDDD}

\pagestyle{empty}

\parskip=1ex
\parindent=0pt

\setlist[enumerate,1]{start=0}

\newcommand\plpage{
    \newpage
}

\newcommand\answerspace{\\\rule[0cm]{0pt}{1cm}}

\newcommand\True{\texttt{True}}
\newcommand\False{\texttt{False}}

\newcommand\startsection[1]{
     \vspace{0.2ex}
    \hrule
    {\fontspec{Oxygen} \tiny
     \vspace{-1ex}
     \emph{#1}
     \vspace{-1.5em}
    }
}

\newfontfamily\headingfont[]{Bree Serif}
\titleformat*{\section}{\LARGE\headingfont}

\begin{document}

\plpage

\section*{System setup for MicroPython ESP32 hacking}

These instructions are for Fedora.
Other systems might be different – try your luck!

\section*{Installation}

\begin{enumerate}

\item Open up a terminal and type:

    \texttt{dnf install picocom ampy}

    If you're not on Fedora and don't have these packages,
    \texttt{screen} would do instead of \texttt{picocom},
    and \texttt{ampy} can be installed via
    \texttt{python3 -m pip install --user adafruit-ampy}.

\item Get yourself to the \texttt{dialout} group.
      You can use:

      \texttt{usermod -a -G \$(whoami)}

      but make sure to login again after that
      (e.g. \texttt{su - \$(whoami)}).
      Verify that you are in the group by running the \texttt{groups} command.

    \textit{
      The \texttt{dialout} group is historically designed for modems
      and gives you full and direct access to serial ports.
    }

\item You need drivers for the on-board CP2102 USB-serial adapter.
    If you are using Linux, you have them.
      If you are not using Linux, may the force be with you
      (ask booth attendants for help).

\end{enumerate}


\section*{Talking to the board}

\textit{
    To protect delicate pins and make connections easier,
    the black ESP32 devkit is plugged into the white OctopusLab Robot~Board.
    Please ask if you want to unplug it.
}

\begin{enumerate}[resume]

\item Connect the ESP32 board via an USB cable.
      Sometimes, the cable or even the board can be faulty,
      use \texttt{dmesg | tail} to verify your system can actually see the board.
      Ask for help if it doesn't.

\item Use \texttt{picocom} to get the interactive interpreter.
      The full command is:

      \texttt{picocom -b 115200 --flow n /dev/ttyUSB0}.

      The first number specifies the baud rate -- a speed of communication over serial line.
      All our ESP32 boards communicate using this baud rate.
      We disable the flow control of the serial-port.
      That should be the default,
      but explicit is better than implicit.

      Lastly, we connect to \texttt{/dev/ttyUSB0}.
      That's the file representing our device (or the serial line to it).
      The filename may be different, consult the \texttt{dmesg | tail} output.

      If you need to use \texttt{screen},
      the command is \texttt{screen /dev/ttyUSB0 115200}.

\item You should see \texttt{Terminal ready}.
      Pres Enter and the MicroPython's interactive prompt with \texttt{>>>} should appear.
      If you see wall of random characters instead,
      locate and press the \emph{EN} button on the device
      (it's near the USB port).

\item Experiment with the interactive Python prompt. Is it Python?
      Can you \texttt{print} stuff? Is something missing?

      If you prefer to code in your favorite text editor,
      you can use \texttt{Ctrl+e} to activate the \emph{paste mode}.

\item To exit \texttt{picocom}, hold the \texttt{Ctrl} key, press \texttt{a},
      then \texttt{q} and lift the \texttt{Ctrl} key.
      Alternatively, unplug the board.

\end{enumerate}

\section*{The board's filesystem}

\begin{enumerate}[resume]

\item The board has a simple filesystem.
      To execute a file on the board when the board starts,
      write a program on your computer and save it as \texttt{main.py}.
      Use \texttt{ampy -p /dev/ttyUSB0 put main.py} to put the file into the board.

      Connect via \texttt{picocom}, then press the \emph{EN} button to see
      what is happening. Use \texttt{Ctrl+c} to interrupt a running program.
      Use \texttt{ampy -p /dev/ttyUSB0 rm main.py} to get rid of the file.
      Explore \texttt{ampy --help} to see what else is possible.

\item Congratulations, you've made it! Now you are ready to pick a task.

    \textit{
    Don't forget to ask for stickers as a reward for your efforts!
    }

\end{enumerate}

\end{document}
