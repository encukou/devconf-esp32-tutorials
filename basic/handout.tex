% arara: lualatex
% arara: lualatex
% Typeset using lualatex

\documentclass[a4paper,10pt]{article}
\usepackage[utf8]{luainputenc}
\usepackage[pdfborder={0 0 0}]{hyperref}

\usepackage[english]{babel}

\usepackage{fouriernc}
\usepackage{tgpagella}
\usepackage{xcolor}
\usepackage[activate=true]{microtype}
\usepackage{amssymb}
\usepackage{fontspec}
\usepackage{titling}
\usepackage{titlesec}

\usepackage{calc}
\usepackage{multicol}
\usepackage{array}
\usepackage{enumitem}
\usepackage{tikz}
\usepackage{siunitx}
\usepackage{colortbl}
\usepackage[a4paper,margin=2cm,top=1cm]{geometry}

\setmainfont[Numbers=OldStyle]{TeX Gyre Pagella}

\definecolor{plhome}{HTML}{CCEEFF}
\definecolor{silver}{HTML}{DDDDDD}

\pagestyle{empty}

\parskip=1ex
\parindent=0pt

\setlist[enumerate,1]{start=0}

\newcommand\plpage{
    \newpage
    \begin{tikzpicture}[remember picture,overlay]
        \node[opacity=0.1,below left,xshift=1cm,yshift=1cm] at (current page.north east) {\includegraphics[width=12cm]{../logo}};
    \end{tikzpicture}
}

\newcommand\answerspace{\\\rule[0cm]{0pt}{1cm}}

\newcommand\True{\texttt{True}}
\newcommand\False{\texttt{False}}

\newcommand\startsection[1]{
     \vspace{0.2ex}
    \hrule
    {\fontspec{Oxygen} \tiny
     \vspace{-1ex}
     \emph{#1}
     \vspace{-1.5em}
    }
}

\newfontfamily\headingfont[]{Bree Serif}
\titleformat*{\section}{\LARGE\headingfont}

\begin{document}

\plpage

\section*{System setup for MicroPython ESP32 hacking}

\begin{enumerate}
\item Get Fedora. If you don't have Fedora, prepare for a slightly worse experience. Open up a terminal.

\end{enumerate}

\startsection{System setup. Use your favorite option to install packages, such as \texttt{dnf install}.}

\begin{enumerate}[resume]

\item \texttt{dnf install picocom ampy}.
      If you don't have such packages in your non-Fedora system,
      \texttt{screen} would do instead of \texttt{picoccom},
      and \texttt{ampy} can be installed via \texttt{python3 -m pip install --user adafruit-ampy}.

\item Get yourself to the \texttt{dialout} group.
      You can use \texttt{usermod -a -G \$(whoami)},
      but make sure to login again after that
      (can be tricked by \texttt{su - \$(whoami)}).
      Verify that you are in the group by running the \texttt{groups} command.
      The \texttt{dialout} group is historically designed for modems
      and gives you full and direct access to serial ports.

\item You need the drivers. If you are using Linux, you have them.
      If you are not using Linux, may the force be with you
      (ask booth attendants for help).

\end{enumerate}


\startsection{Talking to the board.}

\begin{enumerate}[resume]

\item Connect the board with ESP32 (XXX board name?) via an USB cable.
      Sometimes, the cable or even the board can be faulty,
      use \texttt{dmesg | tail} to verify your system can actually see the board.
      Ask for help if it doesn't.

\item Use \texttt{picocom} to get the interactive interpreter.
      The full command is \texttt{picocom -b 115200 --flow n /dev/ttyUSB0}.
      The first number specifies the baud rate -- a speed of communication over serial line,
      all our EPS32 boards communicate using this baud rate.
      We disable the flow control of the serial-port.
      That should be the default,
      but explicit is better than implicit.
      Lastly, we connect to \texttt{/dev/ttyUSB0}.
      That's the file representing our device (or the serial line to it).
      The filename may be different, consult the \texttt{dmesg | tail} output.
      If you need to use \texttt{screen},
      the command is \texttt{screen /dev/ttyUSB0 115200}.

\item You should see \texttt{Terminal ready}.
      Pres Enter and the MicroPython's interactive prompt with \texttt{>>>} should appear.
      If you see wall of random characters instead,
      locate and press the \emph{EN} button on the device
      (it's near the USB port).

\item Experiment with the interactive Python prompt. Is that Python?
      Can you \texttt{print} stuff? Is something missing?
      You can use \texttt{Ctrl+e} to activate the paste mode
      in case you prefer to write the code in your favorite text editor.

\item To exit \texttt{picocom}, hold the \texttt{Ctrl} key, press \texttt{a},
      then \texttt{x} and lift the \texttt{Ctrl} key.
      Alternatively, unplug the board.

\startsection{The board's filesystem.}

\item The board has a simple filesystem.
      To execute a file on the board when the board starts,
      save the file as \texttt{main.py}.
      Use \texttt{ampy -p /dev/ttyUSB0 put main.py} to put the file into the board.
      When you connect via \texttt{picocom}, press the \emph{EN} button to see
      what is happening. Use \texttt{Ctrl+c} to interrupt a running program.
      Use \texttt{ampy -p /dev/ttyUSB0 rm main.py} to get rid of the file.
      Explore \texttt{ampy --help} to see what else is possible.

\item Congratulations, you've made it! Now you are ready to pick a task.
    Don't forget to ask for the Fedora Loves Python sticker as a reward for your efforts.

\end{enumerate}

\end{document}
