% arara: lualatex
% arara: lualatex
% Typeset using lualatex

\documentclass{../tutorial}

\title{Traffic lights with MicroPython on \abbr{ESP32}}

\begin{document}
\begin{enumerate}

\item
    Set up your environment. Ask booth staff for the proper handout.
    You'll need an \abbr{ESP32} board with an \abbr{RGB} \abbr{LED} stripe and a servomotor.

\section{Controlling an \abbr{RGB} \abbr{LED} stripe}

\item
    Trough one of the microcontroller's \emph{pins},
    you are able to control an \abbr{RGB} \abbr{LED} stripe.
    The \abbr{RGB} \abbr{LED} stripe you have consist of 8 cells,
    each containing 3 LEDs: a red one, a green one and a blue one.

    Each of those LEDs can vary in intensity from \lstinline|0| to \lstinline|255|
    and by combining them, you can show a lot of colors.
    In fact you can create so many colors,
    that is has been called the \emph{True Color}.

\item
    MicroPython on the \abbr{ESP32} board already has a high-level class for
    our LED stripe, called \lstinline|NeoPixel|:

    \begin{lstlisting}
    from machine import Pin
    from neopixel import NeoPixel
    \end{lstlisting}

    In order to setup a \lstinline|NeoPixel| instance,
    you'll need to provide an output data pin and the number of cells.

    \begin{lstlisting}
    datapin = Pin(XXX, Pin.OUT)
    np = NeoPixel(datapin, 8)
    \end{lstlisting}

\item
    Write \abbr{RGB} triplets to the object as if it were a list of length 8.
    To make the \abbr{LED}s actually change their color,
    use the \lstinline|np.write()| method:

    \begin{lstlisting}
    np[0] = (255, 112, 68)
    ...
    np[7] = (112, 255, 0)
    np.write()
    \end{lstlisting}

\item
    Get yourself familiarized with the stripe and how is it controlled.
    How do you turn off a cell?

\section{A traffic light}

\item
    Let's design a traffic light.
    First, think how a traffic light looks like:
    It consists of 3 lights, from top to bottom:

    \begin{enumerate}
    \item red,
    \item yellow,
    \item green.
    \end{enumerate}

\item
    Imagine the light sequence. This is what happens in the Czech Republic:

    \begin{enumerate}
    \item red: cars are stopped,
    \item red \& yellow: get ready,
    \item green: cars should go,
    \item yellow: cars should stop.
    \item (repeat)
    \end{enumerate}

    Take in mind that usually the yellow phases are quite shorter than the red and green ones.

\item
    Pick 3 LED cells in a row.
    Decide where is the top and where is the bottom.
    For starters, let's make them all shine in their respective colors.

\item
    Create an infinite loop that simulates a traffic light.
    Use the \lstinline|sleep| function from the \lstinline|time| module:

    \begin{lstlisting}
    from time import sleep
    sleep(1)
    \end{lstlisting}

    Don't forget to turn off the lights that are not used at the moment.

\item
    You've made it!
    Get yourself some swag reward from the booth staff before you turn the page over.

\clearpage

\section{A railroad crossing}

\item
    Let's design a railroad crossing.
    First, think how a traffic light looks like:
    It consists of:

    \begin{itemize}
    \item 2 red lights on the top,
    \item 1 white light on the bottom,
    \item a number of gates (XXX závora? gate?), usually 2 or 4.
    \end{itemize}
      
\item
    Imagine the sequence. This is what happens in the Czech Republic:

    \begin{enumerate}
    \item white blinks, cars, cyclists or pedestrians can go,
    \item white is replaced by blinking reds
          (switching between each other),
          everybody should stop,
    \item the gate(s) go down,
    \item one or more trains pass,
    \item the gate(s) go up,
    \item reds turn off,
    \item (repeat)
    \end{enumerate}

    Take in mind that you need some time before lowering the gate,
    or accidents can happen.

\item
    Pick 3 LED cells.
    Either in a row or get creative and carefully twist the stripe to create 2 rows.
    Decide which cells represent what lights.
    For starters, let's make them all shine in their respective colors.

\item
    Create an infinite loop that simulates the railroad crossing traffic light.

\item
    Let's control the gate.
    You'll use a servomotor as a gate.
    A servomotor can be rotated to a particular angle via
    the hardware \emph{Pulse Width Modulation} (\abbr{PWM}),
    unfortunately there is no higher level library yet.

    Read the \emph{blinking LED} handout for the theory behind PWM.

    Crete a PWM object from an output pin
    with 50 Hz frequency and a duty of 50:

    \begin{lstlisting}
    from machine import Pin, PWM

    servopin = Pin(XXX, Pin.OUT)
    pwm = PWM(servopin, freq=50, duty=50)
    \end{lstlisting}

\item
    To control the angle of your servomotor,
    set duty from 35 (maximum left) trough 77 (the middle) to 120 (maximum right).

    \begin{lstlisting}
    pwm.duty(77)
    \end{lstlisting}

    Get yourself familiarized with controlling the servomotor.

\item
    Add the gate to the railroad crossing's sequence.
    Don't forget to wait between starting the red lights and putting the gate down,
    you don't want anybody to get hurt.

\item
    You rock! Isn't this awesome?

\end{enumerate}
\end{document}
