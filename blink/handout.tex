% arara: lualatex
% arara: lualatex
% Typeset using lualatex

\documentclass{../tutorial}

\title{Blinking LED with MicroPython on \abbr{ESP32}}

\begin{document}

\begin{enumerate}

\item
    You will need a set-up environment and an \ESP board.

\section{Controlling the LED}

\item
    MicroPython allows you to control the microcontroller's \emph{pins},
    to which other devices are connected.
    To do that, you need to import the `Pin` class with:

    \begin{lstlisting}
    from machine import Pin
    \end{lstlisting}

\item
    A pin can serve as a binary input/output.
    On the hardware side, imagine there either is electricity or there isn't.
    On the software side, you can either read `0` or `1` from an
    input pin, or write `0` or `1` to an output pin.

    Create an output Pin object for the integrated LED (\emph{Light Emitting Diode}),
    which is connected to pin \#2, and save it in a variable:

    \begin{lstlisting}
    led = Pin(2, Pin.OUT)
    \end{lstlisting}

\item
    Write 1 or 0 to turn the light on or off:

    \begin{lstlisting}
    led.value(1)
    led.value(0)
    \end{lstlisting}

\section{Blinking the LED with a button}

\item
    Let's create a program that turns the LED on when the user is holding a button.
    We'll use the integrated \emph{BOOT} button, which is connected to pin number 0.
    Create an input Pin for it:

    \begin{lstlisting}
    button = Pin(0, Pin.IN)
    \end{lstlisting}

\item
    Read the value. What is it?

    \begin{lstlisting}
    button.value()
    \end{lstlisting}

\item
    Read the value while holding the \emph{BOOT} button. What is it?

    \begin{comment}
        The \emph{BOOT} button is next to the USB connector.
        Don't confuse it with \emph{EN}!
    \end{comment}

\item
    Using an infinite loop, make the LED shine if and only if you hold the button.

\item
    Change the program so one press of the button switches the light.
    Is there a problem? How can you check for one press?

\section{Blinking the LED with sleep}

\item
    Try out the `sleep` function form the `time` module.
    What does it do?

    \begin{lstlisting}
    from time import sleep
    sleep(1/2)
    \end{lstlisting}

\item
    Make the LED blink 10 times with a half-second interval.

\item
    Make the LED blink really fast.
    On for $\frac{1}{1000}$ of a second, off for $\frac{2}{1000}$ of a second.
    What happens? What if you swap the numbers?

\item
    Fast blinking is a very precise way to control light intensity.
    Most dimmable\footnote{
        Capable of shining at varying intensity, not just off and completely on.
    } lights work exactly this way.
    It's also useful for other peripherals than just lights.

    Since it's so useful, the \ESP had dedicated \emph{hardware \abbr{PWM}}
    circuit for blinking.

    Turn the page to learn how to use it.

\clearpage

\section{Blinking with Pulse Width Modulation}

\item
    If you graph a blinking pin's `value` over time, you get a \emph{square wave}
    like this:

    \begin{figure}[h]
        \centering
        \squarevave{0/1, 3/4, 6/7, 9/10, 12/13}
    \end{figure}

    The integrated hardware \emph{Pulse Width Modulation} (PWM) generates
    such square waves automatically.

    To control it, create a `PWM` class from the `machine` module.
    It takes 3 arguments: a pin, frequency and the duty cycle.

    \begin{lstlisting}
    from machine import PWM
    pwm = PWM(led, freq=50, duty=512)
    \end{lstlisting}

    The frequency (`freq`) determines how fast the LED will blink.
    The duty determines the proportion of time where the LED is on:
    0 means never; 1023 means always (100\%).
    512 means equal time on and off (50\%).

    \begin{comment}
    When experimenting, mind the fact that humans don't perceive light
    intensity linearly.
    A duty cycle of 256 (25\%) is a good “middle” value.
    \end{comment}

    PWM is not blocking -- you can do other things in Python while it's running.
    To change a PWM's frequency or duty, use the `pwm.freq(...)`
    or `pwm.duty(...)` methods.
    To stop a PWM, use `pwm.deinit()`.

    Try it out!

\item
    Please return the board when you're done experimenting.

\end{enumerate}
\end{document}
